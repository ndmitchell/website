% Hoogle-NH.tex
\begin{hcarentry}[updated]{Hoogle}
\label{Hoogle}
\report{Neil Mitchell}%11/10
\status{stable}
\makeheader

Hoogle is an online Haskell API search engine. It searches the functions in the various libraries,
both by name and by type signature. When searching by name, the search just finds functions which
contain that name as a substring. However, when searching by types it attempts to find any functions
that might be appropriate, including argument reordering and missing arguments. The tool is written
in Haskell, and the source code is available online. Hoogle is available as a web interface, a
command line tool, and a lambdabot plugin.

Hoogle development has recently restarted, and work is proceeding quickly. The darcs version of
Hoogle can now search all of Hackage~\cref{hackage}, and should be released in a few months.

\FurtherReading
\url{http://haskell.org/hoogle}
\end{hcarentry}
