% tagsoup-Nt.tex
\begin{hcarentry}{tagsoup}
\label{tagsoup}
\report{Neil Mitchell}%05/10
\makeheader

TagSoup is a library for extracting information out of unstructured HTML code, sometimes
known as tag-soup. The HTML does not have to be well formed, or render properly within
any particular framework. This library is for situations where the author of the HTML
is not cooperating with the person trying to extract the information, but is also not
trying to hide the information.

The library provides a basic data type for a list of unstructured tags, a parser to
convert HTML into this tag type, and useful functions and combinators for finding and
extracting information. The library has seen real use in an application to give
Hackage~\cref{hackagedb} listings, and is used in Hoogle (\url{http://haskell.org/communities/05-2009/html/report.html#sect4.4.1}).

A new version of tagsoup has been released, fully supporting the HTML 5 specification. The
API also has experimental support for ByteString (although currently ByteString is slower
than String).

\FurtherReading
%\begin{compactitem}
%\item 
%Homepage:
\url{http://community.haskell.org/~ndm/tagsoup}
%\end{compactitem}
\end{hcarentry}
